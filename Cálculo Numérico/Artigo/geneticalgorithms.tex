\documentclass[12pt]{article}
\usepackage{sbc-template}
\usepackage{graphicx,url}
\usepackage[brazil]{babel}
\usepackage[utf8]{inputenc}
\usepackage{amssymb}
\usepackage{amsmath}

\sloppy

\title{Otimização Matemática Através de Algoritmos Genéticos}

\author{Fernando Concatto\inst{1}}

\address{Bacharelado em Ciência da Computação -- Universidade do Vale do Itajaí (UNIVALI) \\
  Caixa Postal 360 -- CEP 88302-202 -- Itajaí -- SC -- Brasil
  \email{fernandoconcatto@edu.univali.br}
}

\begin{document}

\maketitle

\begin{abstract}
  Problems involving mathematical optimization are ubiquitous in many fields of science. Due to their elevated computational complexity, several approximation methods have been developed to solve such problems. This paper describes the notion of genetic algorithms, which employs concepts of evolution and natural selection to approximate the highest or lowest value of a function, improving candidate solutions with the passing of generations.
\end{abstract}

\begin{resumo}
  Problemas envolvendo otimização matemática !se mostram ubíquos em diversas áreas da ciência. Por apresentarem uma dificuldade computacional muito elevada, diversos métodos aproximados para resolver tais problemas foram desenvolvidos. Este trabalho descreve os algoritmos genéticos, que aplicam conceitos de evolução e seleção natural para aproximar o ponto máximo ou mínimo de uma função, aprimorando soluções candidatas com o decorrer das gerações.
\end{resumo}


\section{Introdução} \label{sec:intro}

O estudo de problemas e suas soluções compõe o cerne do campo da Ciência da Computação. Para efetuar a solução dos problemas, uma sequência finita de operações deve ser desenvolvida; tais sequências são conhecidas como \textit{algoritmos}. Problemas abordados por cientistas da computação possuem uma enorme gama de formulações e dificuldades, alguns tão complexos que demandam mais do que o tempo de vida um ser humano para serem resolvidos. Um exemplo de problema com esta característica é o Problema do Caixeiro Viajante, definido da seguinte forma: dado um conjunto de $n$ cidades e as distâncias entre elas, qual a rota mais curta que parte de uma cidade, passa por todas as outras e retorna para a cidade inicial? Como existem $(n-1)!/2$ rotas possíveis, descobrir a menor rota entre 30 cidades através de um algoritmo exato requer muito mais do que um milhão de anos de processamento \cite{MacGregor2011}.

Para tratar de problemas com esse nível de complexidade, métodos \textit{heurísticos} são usualmente aplicados. Algoritmos heurísticos são potencialmente capazes de resolver problemas complexos em pouquíssimo tempo, porém não garantem que a solução exata seja encontrada, fornecendo apenas aproximações, que geralmente são suficientes para grande parte dos problemas reais \cite{Kokash2005}. Exemplos de algoritmos heurísticos para problemas específicos incluem o algoritmo de Christofides para o problema do Caixeiro Viajante \cite{Christofides1976}, mencionado anteriormente, e o algoritmo de Kernighan-Lin para particionamento de grafos \cite{Kernighan1970}.

Entretanto, a categoria de algoritmos heurísticos que mais recebeu atenção da comunidade científica é a que envolve a resolução de \textit{problemas de otimização}, caracterizada pela busca por um valor $x \in X$ que maximiza ou minimiza uma função $f: X \rightarrow \mathbb{R}$, geralmente chamada de \textbf{função objetivo}. Grande parte dos problemas? podem ser formulados como problemas de otimização, incluindo o Problema do Caixeiro Viajante, onde a função objetivo pode ser descrita por:

\begin{equation} \label{eq:tsp}
    \mathop{\boldsymbol\min} f(x) = \sum_{i = 1}^{n - 1} d(x_{i}, x_{i + 1}) + d(x_{n}, x_{1})
\end{equation}

\noindent onde $x$ é o conjunto que descreve a rota candidata e $d(x_{i}, x_{j})$ é a distância entre as cidades $x_{i}$ e $x_{j}$. Desta forma, um algoritmo pode mover-se iterativamente pelo \textit{espaço de busca}, que contém todas as rotas possíveis, buscando um conjunto $x$ que minimize a função $f(x)$. Porém, este espaço de busca (e de vários outros problemas) apresenta uma quantidade excessivamente grande de elementos, exigindo então a aplicação de métodos heurísticos.

Problemas de otimização envolvem várias disciplinas, como pesquisa operacional, economia e as ciências naturais. A análise numérica é um dos campos com maior influência no estudo dos problemas de otimização, devido ao foco em métodos iterativos e aproximados, como o Método de Newton, através do qual é possível encontrar máximos ou mínimos locais se for aplicado à primeira derivada de uma função. Outro método capaz de encontrar pontos críticos é o Método dos Gradientes, que guia uma solução candidata para um ponto mínimo ou máximo a partir do gradiente de uma função. Este método é frequentemente aplicado em Redes Neurais Artificiais para guiar a aprendizagem da rede \cite{Haykin1998}.

Este trabalho busca descrever o método heurístico conhecido como Algoritmo Genético, que utiliza conceitos da biologia evolucionária para otimizar uma função objetivo. Para este fim, a seção \ref{sec:evolution} introduz alguns conceitos relativos à evolução por seleção natural; por sua vez, a seção \ref{sec:algorithm} apresenta a construção do Algoritmo Genético, descrevendo sua estrutura fundamental; já a seção \ref{sec:operators} define os operadores genéticos de seleção, mutação e reprodução; por fim, a seção \ref{sec:conclusions} apresenta uma breve discussão sobre o método, concluindo o trabalho.

\section{O processo evolucionário} \label{sec:evolution}

O conceito de evolução de organismos é bastante antigo, datando da época dos filósofos gregos pré-socráticos \cite{Hull1967}. A noção de seleção natural  foi desenvolvida por Charles Darwin, com a publicação de seu livro ``A Origem das Espécies'', de 1859. Nele, Darwin 

\section{O algoritmo genético} \label{sec:algorithm}

Lorem ipsum dolor sit amet, consectetur adipisicing elit, sed do eiusmod tempor incididunt ut labore et dolore magna aliqua. Ut enim ad minim veniam, quis nostrud exercitation ullamco laboris nisi ut aliquip ex ea commodo consequat. Duis aute irure dolor in reprehenderit in voluptate velit esse cillum dolore eu fugiat nulla pariatur. Excepteur sint occaecat cupidatat non proident, sunt in culpa qui officia deserunt mollit anim id est laborum.

\section{Operadores genéticos} \label{sec:operators}

Lorem ipsum dolor sit amet, consectetur adipisicing elit, sed do eiusmod tempor incididunt ut labore et dolore magna aliqua. Ut enim ad minim veniam, quis nostrud exercitation ullamco laboris nisi ut aliquip ex ea commodo consequat. Duis aute irure dolor in reprehenderit in voluptate velit esse cillum dolore eu fugiat nulla pariatur. Excepteur sint occaecat cupidatat non proident, sunt in culpa qui officia deserunt mollit anim id est laborum.

\section{Discussão e conclusões} \label{sec:conclusions}

Lorem ipsum dolor sit amet, consectetur adipisicing elit, sed do eiusmod tempor incididunt ut labore et dolore magna aliqua. Ut enim ad minim veniam, quis nostrud exercitation ullamco laboris nisi ut aliquip ex ea commodo consequat. Duis aute irure dolor in reprehenderit in voluptate velit esse cillum dolore eu fugiat nulla pariatur. Excepteur sint occaecat cupidatat non proident, sunt in culpa qui officia deserunt mollit anim id est laborum.

\bibliographystyle{sbc}
\bibliography{bibliography}

\end{document}
