\documentclass[12pt]{article}
\usepackage{sbc-template}
\usepackage{graphicx,url}
\usepackage[brazil]{babel}
\usepackage[utf8]{inputenc}
\usepackage{amssymb}
\usepackage{amsmath}
\usepackage{comment}
\usepackage[linesnumbered,ruled,vlined,portuguese]{algorithm2e}

\sloppy

\title{Otimização Matemática Através de Algoritmos Genéticos}

\author{Fernando Concatto\inst{1}}

\address{Bacharelado em Ciência da Computação -- Universidade do Vale do Itajaí (UNIVALI) \\
  Caixa Postal 360 -- CEP 88302-202 -- Itajaí -- SC -- Brasil
  \email{fernandoconcatto@edu.univali.br}
}

\begin{document}

\maketitle

\begin{abstract}
  Problems involving mathematical optimization are ubiquitous in many fields of science. Due to their elevated computational complexity, several approximation methods have been developed to solve such problems. This paper describes the notion of genetic algorithms, which employs concepts of evolution and natural selection to approximate the highest or lowest value of a function, improving candidate solutions with the passing of generations.
\end{abstract}

\begin{resumo}
  Problemas envolvendo otimização matemática !se mostram ubíquos em diversas áreas da ciência. Por apresentarem uma dificuldade computacional muito elevada, diversos métodos aproximados para resolver tais problemas foram desenvolvidos. Este trabalho descreve os algoritmos genéticos, que aplicam conceitos de evolução e seleção natural para aproximar o ponto máximo ou mínimo de uma função, aprimorando soluções candidatas com o decorrer das gerações.
\end{resumo}


\section{Introdução} \label{sec:intro}

O estudo de problemas e suas soluções compõe o cerne do campo da Ciência da Computação. Para efetuar a solução dos problemas, uma sequência finita de operações deve ser desenvolvida; tais sequências são conhecidas como \textit{algoritmos}. Problemas abordados por cientistas da computação possuem uma enorme gama de formulações e dificuldades, alguns tão complexos que demandam mais do que o tempo de vida um ser humano para serem resolvidos. Um exemplo de problema com esta característica é o Problema do Caixeiro Viajante, definido da seguinte forma: dado um conjunto de $n$ cidades e as distâncias entre elas, qual a rota mais curta que parte de uma cidade, passa por todas as outras e retorna para a cidade inicial? Como existem $(n-1)!/2$ rotas possíveis, descobrir a menor rota entre 30 cidades através de um algoritmo exato requer muito mais do que um milhão de anos de processamento \cite{MacGregor2011}.

Para tratar de problemas com esse nível de complexidade, métodos \textit{heurísticos} são usualmente aplicados. Algoritmos heurísticos são potencialmente capazes de resolver problemas complexos em pouquíssimo tempo, porém não garantem que a solução exata seja encontrada, fornecendo apenas aproximações, que geralmente são suficientes para grande parte dos problemas reais \cite{Kokash2005}. Exemplos de algoritmos heurísticos para problemas específicos incluem o algoritmo de Christofides para o problema do Caixeiro Viajante \cite{Christofides1976}, mencionado anteriormente, e o algoritmo de Kernighan-Lin para particionamento de grafos \cite{Kernighan1970}.

Entretanto, a categoria de algoritmos heurísticos que mais recebeu atenção da comunidade científica é a que envolve a resolução de \textit{problemas de otimização}, caracterizada pela busca por um valor $x \in X$ que maximiza ou minimiza uma função $f: X \rightarrow \mathbb{R}$, geralmente chamada de \textbf{função objetivo}. Grande parte dos problemas? podem ser formulados como problemas de otimização, incluindo o Problema do Caixeiro Viajante, onde a função objetivo pode ser descrita por:

\begin{equation} \label{eq:tsp}
    \mathop{\boldsymbol\min} f(x) = \sum_{i = 1}^{n - 1} d(x_{i}, x_{i + 1}) + d(x_{n}, x_{1})
\end{equation}

\noindent onde $x$ é o conjunto que descreve a rota candidata e $d(x_{i}, x_{j})$ é a distância entre as cidades $x_{i}$ e $x_{j}$. Desta forma, um algoritmo pode mover-se iterativamente pelo \textit{espaço de busca}, que contém todas as rotas possíveis, buscando um conjunto $x$ que minimize a função $f(x)$. Porém, este espaço de busca (e de vários outros problemas) apresenta uma quantidade excessivamente grande de elementos, exigindo então a aplicação de métodos heurísticos.

Problemas de otimização envolvem várias disciplinas, como pesquisa operacional, economia e as ciências naturais. A análise numérica é um dos campos com maior influência no estudo dos problemas de otimização, devido ao foco em métodos iterativos e aproximados, como o Método de Newton, através do qual é possível encontrar máximos ou mínimos locais se for aplicado à primeira derivada de uma função. Outro método capaz de encontrar pontos críticos é o Método dos Gradientes, que guia uma solução candidata para um ponto mínimo ou máximo a partir do gradiente de uma função. Este método é frequentemente aplicado em Redes Neurais Artificiais para guiar a aprendizagem da rede \cite{Haykin1998}.

Este trabalho busca descrever o método heurístico conhecido como Algoritmo Genético, que utiliza conceitos da biologia evolucionária para otimizar uma função objetivo. Para este fim, a seção \ref{sec:introevolution} introduz alguns conceitos relativos à evolução por seleção natural; por sua vez, a seção \ref{sec:algorithm} apresenta a construção do Algoritmo Genético, descrevendo sua estrutura fundamental; já a seção \ref{sec:geneticoperators} define os operadores genéticos de seleção, mutação e reprodução; por fim, a seção \ref{sec:conclusions} apresenta uma breve discussão sobre o método, concluindo o trabalho.

\section{O processo evolucionário} \label{sec:introevolution}

O conceito de evolução de organismos é bastante antigo, datando da época dos filósofos gregos pré-socráticos \cite{Hull1967}. A noção de seleção natural  foi desenvolvida por Charles Darwin, com a publicação de seu livro ``A Origem das Espécies'', de 1859. Nele, Darwin propõe que populações evoluem ao longo de gerações através de um processo de seleção que favorece organismos com aptidão alta, medida através do sucesso reprodutivo. Indivíduos aptos produzem descendentes, que herdam boa parte de suas características (genes), e portanto também tendem a possuir elevado sucesso reprodutivo.

Desta maneira, a cada geração, os melhores indivíduos da população são escolhidos naturalmente para reprodução, passando adiante suas características, enquanto os indivíduos menos aptos tendem a desaparecer. Assim, a próxima geração será composta tanto pelos indivíduos aptos sobreviventes quanto por seus descendentes, ambos possuindo uma aptidão média superior à geração passada. Algoritmos genéticos buscam simular esse processo, modelando os mecanismos da natureza de forma computacional, com o intuito de otimizar uma função objetivo.

\section{Os algoritmos genéticos} \label{sec:algorithm}

Algoritmos genéticos são classificados como meta-heurísticas, ou seja, são independentes do problema sendo resolvido. Qualquer problema que possa ser formulado por uma função objetivo pode ser solucionado por um algoritmo genético. Adicionalmente, algoritmos genéticos não possuem uma definição absolutamente rigorosa quanto aos detalhes dos procedimentos realizados; ao invés disso, são caracterizados pela existência de alguns elementos fundamentais \cite{Mitchell1998}, vistos adiante.

\subsection{Modelagem de cromossomos} \label{sec:chromosomes}

O primeiro destes elementos é a especificação dos \textbf{cromossomos}, que caracterizarão os indivíduos população?. Cromossomos representam as soluções do problema, e a maneira como eles são codificados é um dos principais fatores envolvidos no desenvolvimento de um algoritmo genético bem sucedido.

Historicamente, o método mais utilizado para modelar cromossomos foi a codificação por cadeias de bits, porém outros métodos podem ser empregados, especialmente para problemas mais complexos \cite{Mitchell1998}. Por exemplo, para o Problema do Caixeiro Viajante, apresentado na introdução deste trabalho, um cromossomo pode ser caracterizado pela sequência de cidades que especifica a rota a ser tomada.

\subsection{Função de aptidão} \label{sec:fitnessfunction}

O segundo elemento fundamental de um algoritmo genético é a função objetivo propriamente dita, denominada \textbf{função de aptidão} nesse contexto. A função de aptidão é responsável por avaliar a qualidade dos indivíduos, e é utilizada para selecioná-los para reprodução. Geralmente, devido ao caráter biológico, a função de aptidão é maximizada; caso a intenção seja a minimizar a função, é possível utilizar o inverso multiplicativo da função, definido como $1 / f(x)$.

\subsection{Operadores genéticos} \label{sec:operatorsbrief}

Por fim, os elementos que concluem a base dos algoritmos genéticos são os \textbf{operadores genéticos}. Os principais operadores são a \textbf{seleção}, que utiliza a função de aptidão para selecionar indivíduos; a \textbf{reprodução}, que combina dois indivíduos para produzir um descendente; e a \textbf{mutação}, que aleatoriamente altera um ou mais genes do cromossomo. Estes operadores e as técnicas utilizadas para implementá-los são detalhados na seção \ref{sec:geneticoperators}.

\subsection{Evolução} \label{sec:evolution}

Com todos estes elementos definidos, resta especificar o procedimento de evolução propriamente dito. Este é um dos segmentos mais flexíveis de um algoritmo genético. Muitas variações existem para a implementação deste processo, algumas extremamente complexas e outras razoavelmente simples. Um exemplo básico, porém bastante eficiente para grande parte dos problemas, é apresentado adiante.

Inicialmente, uma população de $n$ indivíduos é estabelecida, com seus cromossomos construídos aleatoriamente ou através de outros métodos.
Como a otimização por algoritmos genéticos é um processo iterativo, os seguintes passos são repetidos até que uma solução satisfatória seja encontrada. A primeira etapa consiste no cálculo da aptidão de cada indivíduo $x$ da população, através da função de aptidão $f(x)$. Na sequência, uma nova geração de indivíduos é produzida a partir da aplicação dos operadores genéticos.

Dois indivíduos são selecionados através do operador de seleção, que favorece indivíduos mais aptos. Então, um ou mais filhos são concebidos pelo operador de reprodução, herdando as características de seus pais. Cada filho passa pelo operador de mutação, possivelmente sofrendo alterações genéticas em seu cromossomo. Este procedimento é repetido até que uma quantidade suficiente de filhos seja gerada.

Por fim, a população de indivíduos que foram selecionados para reprodução é unida com a população de filhos concebidos, formando uma nova população de tamanho $n$. Este processo é repetido até que alguma determinada condição de término seja atingida. O algoritmo \ref{alg:geneticalgorithm} formaliza essa sequência de passos por meio de uma representação em pseudocódigo.

\begin{algorithm}[ht]
  \small
  \DontPrintSemicolon
  \caption{Algoritmo genético básico}
  \label{alg:geneticalgorithm}
  \SetKwInOut{Input}{Entrada}
  \SetKwInOut{Output}{Saída}
  \Input{Tamanho da população ($n$), chance de mutação ($m$)}
  \Output{Indivíduos evoluídos}
  \BlankLine
  $populacao \leftarrow gerarPopulacaoInicial(n)$ \;
  \BlankLine
  \Repeat{condição de parada não for atingida}{
      \ForEach{individuo $\in$ populacao}{
        $individuo.aptidao \leftarrow calcularAptidao(individuo)$ \;
      }
      \BlankLine
      $filhos \leftarrow \emptyset$ \;
      \BlankLine
      \While{quantidade de filhos for insuficiente}{
        $pais \leftarrow selecionar(populacao)$ \;
        $filho \leftarrow reproduzir(pais)$ \;
        \BlankLine
        \If{aleatorio() $<$ m}{
            $aplicarMutacao(filho)$ \;
        }
        \BlankLine
        $filhos \leftarrow filhos \cup \{filho\}$ \;
      }
      $populacao \leftarrow combinar(populacao, filhos)$ \;
  }
  \BlankLine
  \Return{populacao}
\end{algorithm}

Ao término deste algoritmo, caso não haja nenhuma falha na definição dos cromossomos, da função de aptidão e dos operadores genéticos, a população final possuirá uma qualidade consideravelmente superior à inicial, dependendo da topologia do espaço de busca do problema e da quantidade de gerações decorridas.

\section{Operadores genéticos} \label{sec:geneticoperators}

Falar sobre operadores genéticos!

\section{Aplicações dos algoritmos genéticos} \label{sec:applications}

Falar sobre as aplicações

\section{Discussão e conclusões} \label{sec:conclusions}

Lorem ipsum dolor sit amet, consectetur adipisicing elit, sed do eiusmod tempor incididunt ut labore et dolore magna aliqua. Ut enim ad minim veniam, quis nostrud exercitation ullamco laboris nisi ut aliquip ex ea commodo consequat. Duis aute irure dolor in reprehenderit in voluptate velit esse cillum dolore eu fugiat nulla pariatur. Excepteur sint occaecat cupidatat non proident, sunt in culpa qui officia deserunt mollit anim id est laborum.

\bibliographystyle{sbc}
\bibliography{bibliography}

\end{document}
